\documentclass{report}

%-----------------------------------
%--- Hugoooo default latex header---
%-----------------------------------

%---PACKAGES---

\usepackage[Glenn]{fncychap}

\usepackage{fancyhdr}

\usepackage[utf8x]{inputenc} 
\usepackage[T1]{fontenc}      
\usepackage[french]{babel} 

\usepackage{array}

\usepackage{mathtools}
\usepackage{amssymb}
\usepackage{mathrsfs}
\usepackage{mathabx}

\usepackage{listings}
\usepackage{xcolor}
\usepackage{graphicx}

\usepackage[a4paper]{geometry}
\geometry{hscale=0.85,vscale=0.85,centering}

%%\BB
\newcommand{\QQ}{\mathbb{Q}}
\newcommand{\WW}{\mathbb{W}}
\newcommand{\EE}{\mathbb{E}}
\newcommand{\RR}{\mathbb{R}}
\newcommand{\TT}{\mathbb{T}}
\newcommand{\YY}{\mathbb{Y}}
\newcommand{\UU}{\mathbb{U}}
\newcommand{\II}{\mathbf{1}}
\newcommand{\OO}{\mathbb{O}}
\newcommand{\PP}{\mathbb{P}}
\renewcommand{\AA}{\mathbb{A}}
\renewcommand{\SS}{\mathbb{S}}
\newcommand{\DD}{\mathbb{D}}
\newcommand{\FF}{\mathbb{F}}
\newcommand{\GG}{\mathbb{G}}
\newcommand{\HH}{\mathbb{H}}
\newcommand{\JJ}{\mathbb{J}}
\newcommand{\KK}{\mathbb{K}}
\newcommand{\LL}{\mathbb{L}}
\newcommand{\ZZ}{\mathbb{Z}}
\newcommand{\XX}{\mathbb{X}}
\newcommand{\CC}{\mathbb{C}}
\newcommand{\VV}{\mathbb{V}}
\newcommand{\BB}{\mathbb{B}}
\newcommand{\NN}{\mathbb{N}}
\newcommand{\MM}{\mathbb{M}}


%%\CAL
\newcommand{\A}{\mathcal{A}}
\newcommand{\B}{\mathcal{B}}
\newcommand{\C}{\mathcal{C}}
\newcommand{\D}{\mathcal{D}}
\newcommand{\E}{\mathcal{E}}
\newcommand{\F}{\mathcal{F}}
\newcommand{\G}{\mathcal{G}}
\renewcommand{\H}{\mathcal{H}}
\newcommand{\I}{\mathcal{I}}
\newcommand{\J}{\mathcal{J}}
\newcommand{\K}{\mathcal{K}}
\renewcommand{\L}{\mathcal{L}}
\newcommand{\M}{\mathcal{M}}
\newcommand{\N}{\mathcal{N}}
\renewcommand{\O}{\mathcal{O}}
\renewcommand{\P}{\mathcal{P}}
\newcommand{\Q}{\mathcal{Q}}
\newcommand{\R}{\mathcal{R}}
\renewcommand{\S}{\mathcal{S}}
\newcommand{\T}{\mathcal{T}}
\newcommand{\U}{\mathcal{U}}
\newcommand{\V}{\mathcal{V}}
\newcommand{\W}{\mathcal{W}}
\newcommand{\X}{\mathcal{X}}
\newcommand{\Y}{\mathcal{Y}}
\newcommand{\Z}{\mathcal{Z}}

%---COMMANDS---
\newcommand{\function}[5]{\begin{array}[t]{lrcl}
#1: & #2 & \longrightarrow & #3 \\
    & #4 & \longmapsto & #5 \end{array}}
    
\newcommand{\vect}{\text{vect}}
\renewcommand{\ker}{\text{Ker}}
\newcommand{\ens}[3]{\mathcal{#1}_{#2}(\mathbb{#3})}
\newcommand{\ensmat}[2]{\ens{M}{#1}{#2}}
\newcommand{\mat}{\text{Mat}}
\newcommand{\comp}{\text{Comp}}
\newcommand{\pass}{\text{Pass}}
\renewcommand{\det}{\text{det}}
\newcommand{\dev}{\text{Dev}}
\newcommand{\com}{\text{Com}}
\newcommand{\card}{\text{card}}
\newcommand{\esc}{\text{Esc}}
\newcommand{\cpm}{\text{CPM}}
\newcommand{\dif}{\mathop{}\!\textnormal{\slshape d}}
\newcommand{\enc}[3]{\left#1 #2 \right#3}
\newcommand{\ent}[2]{\enc{#1}{#2}{#1}}
\newcommand{\norm}[1]{\ent{\|}{#1}}
\newcommand{\pth}[1]{\left( #1 \right)}
\newcommand{\trans}{\text{Trans}}
\newcommand{\ind}{\text{Ind}}
\newcommand{\tfinite}{\text{T-finite}}
\newcommand{\tinfinite}{\text{T-infinite}}

\lstset{ % general style for listings
   numbers=left
   , tabsize=2
   , frame=single
   , breaklines=true
   , basicstyle=\ttfamily
   , numberstyle=\tiny\ttfamily
   , framexleftmargin=13mm
   , xleftmargin=12mm
   %, frameround={tttt}
   , captionpos=b
}
\lstdefinestyle{xslt}
{
    emph={xsl,template,variable,param,for,each,apply,templates,with,param}
    , emph={[2]match, select, name, mode}
}
%---HEAD---

\title{}
\author{}
\date{\today}

\renewcommand\thesection{\arabic{section}}

\pagestyle{fancy}
\fancyhf{}
\lhead{}
\chead{Projet Prog 1}
\rhead{Hugo Fruchet}
\cfoot{\thepage}

\begin{document}

\section{Implémentation de la liste}

{\small my\_list.ml}
\begin{lstlisting}[language=caml, style=xslt]
type 'a my_list = 
  | Nil
  | Cons of 'a * 'a my_list

let string_of_list f l =
  let rec aux = function
    | Nil -> ""
    | Cons (h,Nil) -> f h
    | Cons (h,t) -> (f h) ^ ", " ^ (aux t)
  in "[" ^ (aux l) ^ "]"

let hd = function
  | Nil -> None
  | Cons (h,_) -> Some h

let tl = function
  | Nil -> None
  | Cons (_,t) -> Some t

let rec length = function
  | Nil -> 0
  | Cons (_,t) -> 1 + (length t)

let rec map f = function
  | Nil -> Nil
  | Cons (h,t) -> Cons (f h, map f t)
\end{lstlisting}

{\small test\_my\_list.ml}
\begin{lstlisting}[language=caml, style=xslt]
open Option;;
open My_list;;

let string_of_nat_list = string_of_list string_of_int in
let string_of_string_list = string_of_list (fun x -> x) in

let empty = Nil in
let one = Cons ("a", Nil) in 
let lst = Cons (1, Cons (3, Cons (6, Cons (10, Cons (15, Cons (21, Cons (28, Cons (36, Cons (45, Cons (55, Nil)))))))))) in

let test_hd () = 
  Printf.printf "Tete de %s : %s.\n" (string_of_string_list one) (get (hd one));
  Printf.printf "Tete de %s : %d.\n\n" (string_of_nat_list lst) (get (hd lst))

in let test_tl () = 
  Printf.printf "Queue de %s : %s.\n" (string_of_string_list one) (string_of_string_list (get (tl one)));
  Printf.printf "Queue de %s : %s.\n\n" (string_of_nat_list lst) (string_of_nat_list (get (tl lst)))

in let test_length () = 
  Printf.printf "Taille de %s : %d.\n" (string_of_string_list one) (length one);
  Printf.printf "Taille de %s : %d.\n" (string_of_nat_list lst) (length lst);
  Printf.printf "Taille de %s : %d.\n\n" (string_of_string_list empty) (length empty)

in let test_map ()= 
  Printf.printf "Map de (x -> xx) sur %s : %s.\n" (string_of_string_list one) (string_of_string_list (map (fun s -> s ^ s) one));
  Printf.printf "Map de (x -> 2x) sur %s : %s.\n" (string_of_nat_list lst) (string_of_nat_list (map (fun n -> 2 * n) lst));
  Printf.printf "Map de (x -> 2x) sur %s : %s.\n\n" (string_of_nat_list empty) (string_of_nat_list (map (fun n -> 2 * n) empty));

in test_hd(); test_tl(); test_length(); test_map()
\end{lstlisting}

\section{Test du module}
\begin{lstlisting}[style=xslt]
Tete de [a] : a.
Tete de [1, 3, 6, 10, 15, 21, 28, 36, 45, 55] : 1.

Queue de [a] : [].
Queue de [1, 3, 6, 10, 15, 21, 28, 36, 45, 55] : [3, 6, 10, 15, 21, 28, 36, 45, 55].

Taille de [a] : 1.
Taille de [1, 3, 6, 10, 15, 21, 28, 36, 45, 55] : 10.
Taille de [] : 0.

Map de (x -> xx) sur [a] : [aa].
Map de (x -> 2x) sur [1, 3, 6, 10, 15, 21, 28, 36, 45, 55] : [2, 6, 12, 20, 30, 42, 56, 72, 90, 110].
Map de (x -> 2x) sur [] : [].
\end{lstlisting}

\end{document}
